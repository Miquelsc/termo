\documentclass[12pt]{article}

\usepackage[catalan]{babel}
\usepackage[utf8]{inputenc}
\usepackage[T1]{fontenc}
\usepackage{lmodern}
\usepackage{multirow}
\usepackage{graphicx}
\usepackage[a4paper,total={6.2in,9.3in}]{geometry}
\usepackage{pdflscape}
% Formatting options
\usepackage[bf,sf]{titlesec} % Make the section titles bold and sans-serif
\usepackage[font={footnotesize, sf}, labelfont=bf]{caption} % Format dels peus de figura
\renewcommand{\arraystretch}{1.7}
\usepackage{amsmath,amssymb}
\usepackage{amsmath}

\newcommand{\abs}[1]{\left\lvert#1\right\rvert}
\newcommand{\R}{\mathbb{R}}
\newcommand{\N}{\mathbb{N}}

\newcommand{\yestag}{\refstepcounter{equation}\tag{\theequation}}

\title{\sffamily {\bfseries EDIM I:} Informe calefacció}
\author{\sffamily Raquel García Bellés, Miquel Saucedo Cuesta}
\date{}

\begin{document}
	
\maketitle
\section*{Introducció}
\begin{equation}\label{principal}
	T'=q(t)-k(T-T_e(t)),\quad T(0)=T_0
\end{equation}
\section*{Supòsits}
\begin{equation}\label{te}
	T_e(t)=\frac{T_{max}+T_{min}}{2}+\frac{T_{max}-T_{min}}{2}\sin\left(\omega t+\pi\right)
\end{equation}
\section*{Models}
	\subsection{Model 1}
	\subsection{Model 2}
	En aquest model considerem una $q(t)$ que depén de com estigui variant la temperatura de la casa, segons:
	\begin{equation}\label{model2_1}
		q(t)=\alpha T'(t)
	\end{equation}
	on $\alpha$ és una constant. Observem que introduïnt \eqref{model2_1} en \eqref{principal} obtenim:
	\begin{equation}
		T'=-\frac{k}{1-\alpha}(T-T_e(t))
	\end{equation}
	aleshores si anomenem $k'=k/(1-\alpha)$, aquest model és equivalent a poder canviar la constant $k$.
	El gasto energètic d'aquest model en un periòde de $24$h vindrà donat per:
	\begin{equation}
		\int_{0}^{24}\alpha T'dt=\alpha\Delta T
	\end{equation}
\section*{Conclusions}
\section*{Annex}
\end{document}